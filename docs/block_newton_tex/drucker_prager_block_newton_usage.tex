\documentclass[a4paper,11pt]{article}
\usepackage{amsmath,amssymb}
\usepackage{algorithm}
\usepackage{algpseudocode}
\usepackage{booktabs}
\usepackage{geometry}
\usepackage{siunitx}
\usepackage{hyperref}
\geometry{margin=25mm}
\newcommand{\tensor}[1]{\boldsymbol{#1}}
\newcommand{\bsigma}{\tensor{\sigma}}
\newcommand{\beps}{\tensor{\varepsilon}}
\newcommand{\bC}{\tensor{\mathsf{C}}}
\newcommand{\bB}{\tensor{\mathsf{B}}}
\newcommand{\bL}{\tensor{\mathsf{L}}}
\newcommand{\bM}{\tensor{\mathsf{M}}}
\newcommand{\bN}{\tensor{\mathsf{N}}}
\newcommand{\bn}{\tensor{n}}
\title{Block Newton Usage Guide for the Drucker--Prager Model in \texttt{plstss240}}
\author{}
\date{}
\begin{document}
\maketitle
\section{Overview}
The Drucker--Prager plasticity update based on the block Newton method is implemented in four cooperating Fortran routines:
\begin{itemize}
  \item the global Newton loop in \texttt{analys.f}, which repeatedly assembles residuals and tangents before checking convergence \cite{analysNR};
  \item the element post-processing driver \texttt{postpr.f}, which dispatches to element routines while forwarding the global iteration counter and the history workspace \cite{postprDispatch};
  \item the eight-node hexahedral element routine \texttt{phexa8.f}, responsible for evaluating strains, managing the history vectors, and calling the constitutive update according to the material type flag \cite{phexa8Call};
  \item the integration-point algorithm \texttt{stress\_dp\_bn.f}, which realises BOX~1 (local update) and the storage expected by BOX~2 (global update) of Yamamoto et~al.~(2021) \cite{stressBNBox1,stressBNHistory}.
\end{itemize}
This document summarises how these routines exchange data and provides algorithm listings that follow standard notation.
\section{File responsibilities and data flow}
\subsection{Global Newton loop (\texttt{analys.f})}
For each load step, \texttt{analys.f} solves the linearised equilibrium equations, updates the displacement increment, and invokes \texttt{postpr} to obtain the internal force vector \cite{analysNR}. The call passes the current Newton iteration number \(\mathrm{itr}\) and the integration-point work array \(\texttt{histi0}(50, n_{\mathrm{gp}}, n_{\mathrm{el}})\), which stores block Newton residual information between iterations.
\subsection{Element post-processing driver (\texttt{postpr.f})}
Within \texttt{postpr}, the mesh is traversed element-by-element. For each eight-node hexahedron the routine gathers coordinates and displacements and calls \texttt{phexa8} with the same iteration counter and the history array slices \cite{postprDispatch}. This keeps the history data synchronised with the global loop.
\subsection{Element routine (\texttt{phexa8.f})}
Inside \texttt{phexa8} the strain tensor is computed at every Gauss point, the history arrays \(\texttt{ehist}\) and \(\texttt{histi}\) are loaded, and the constitutive model is selected by the material flag \texttt{MATYPE} \cite{phexa8Call}. The branch that currently calls \texttt{stress\_dp\_rm} (return mapping) for \texttt{MATYPE=4} can be switched to \texttt{stress\_dp\_bn} to activate the block Newton update. After the constitutive call, updated history values and consistent tangents are written back to \texttt{histi0} for use in the next global iteration.
\subsection{Integration-point algorithm (\texttt{stress\_dp\_bn.f})}
The routine \texttt{stress\_dp\_bn} loads the stored plastic strain tensor and equivalent plastic strain, distinguishes the first global iteration from subsequent ones, and evaluates the Drucker--Prager trial state \cite{stressBNBox1}. For plastic loading the subroutine executes the BOX~1 update: it reconstructs the incrementally consistent plastic strain, updates the hardening variables, forms the matrices \(\bL\), \(\bM\), and the scalar \(N\), and computes the corrected stress and consistent tangent \cite{stressBNUpdate}. Key scalar residuals (yield function \(g\), consistency increment \(\delta\gamma\)) and sensitivities (\(\partial g/\partial \beps\)) are stored back into \texttt{histi} so that the global loop can assemble BOX~2 \cite{stressBNHistory}. In elastic steps the routine returns the elastic stress and tangent while zeroing the residual data \cite{stressBNElastic}.
\section{Algorithm listings}
\subsection{Global block Newton iteration}
\begin{algorithm}[H]
  \caption{Global block Newton loop (\texttt{analys.f})}
  \label{alg:global}
  \begin{algorithmic}[1]
    \State Initialise displacement vectors, history arrays, and load factors.
    \For{each load step}
      \Repeat
        \State Assemble and solve the linearised equilibrium equations.
        \State Update the displacement field \(\mathbf{u}^{k+1} = \mathbf{u}^{k} + \delta \mathbf{u}\).
        \State Call \texttt{postpr} to evaluate internal forces and collect block Newton data (Alg.~\ref{alg:element}).
        \State Form the residual \(\mathbf{R}_f = \mathbf{F}_{\mathrm{ext}} - \mathbf{F}_{\mathrm{int}}\) and compute its norm.
      \Until the residual and load-normalised norms satisfy the tolerance.
      \State Store converged histories and advance to the next load factor.
    \EndFor
  \end{algorithmic}
\end{algorithm}
\subsection{Element assembly and residual accumulation}
\begin{algorithm}[H]
  \caption{Element loop for hexahedra (\texttt{postpr.f}/\texttt{phexa8.f})}
  \label{alg:element}
  \begin{algorithmic}[1]
    \For{each element \(e\)}
      \State Gather nodal coordinates and displacements.
      \For{each Gauss point \(g\)}
        \State Compute strains \(\beps_{g}\) and load history slices.
        \State Invoke \texttt{stress\_dp\_bn} to obtain \(\bsigma_g\), \(\bC^{\mathrm{ep}}_g\), and updated history data (Alg.~\ref{alg:local}).
        \State Accumulate internal forces \(\mathbf{f}_{\mathrm{int}} += \int_{\Omega_e} \bB^T \bsigma_g\,\mathrm{d}\Omega\).
        \State Store the returned history back into \texttt{histi0} and \texttt{dhist1}.
      \EndFor
      \State Assemble element contributions into the global residual and tangent matrices.
    \EndFor
  \end{algorithmic}
\end{algorithm}
\subsection{Integration-point update and storage}
\begin{algorithm}[H]
  \caption{BOX~1 at a Gauss point (\texttt{stress\_dp\_bn.f})}
  \label{alg:local}
  \begin{algorithmic}[1]
    \State Load \(\alpha_n\), \(\beps_n^{\mathrm{p}}\), and previously stored block Newton data from \texttt{histi}.
    \If{first global iteration}
      \State Set \(\delta\gamma=0\) and evaluate the trial yield function \(g^{\mathrm{tr}}\).
      \State Initialise \(\bL\), \(\bM\), \(N\), and store the residual \(g=g^{\mathrm{tr}}\).
    \Else
      \State Recover the increment \(\delta\gamma\) supplied through \texttt{histi(1)} and rebuild the updated plastic strain.
      \State Update the hardening function \(H(\alpha)\) and the yield function \(g\).
      \State Recompute \(\bL\), \(\bM\), and \(N\) for the new state.
    \EndIf
    \State Form the corrected stress \(\bsigma = \bsigma^{\mathrm{tr}} - \sqrt{2}\mu\,\delta\gamma\,\bn + p\,\tensor{I}\) and add the consistency correction \(-gN^{-1}\bL\).
    \State Assemble the consistent tangent \(\bC^{\mathrm{ep}} = \bC - N^{-1} \bL \otimes \bM\).
    \State Store \(g\), \(N\), the strain sensitivities \(\partial g/\partial \beps\), and \(\bM\) into \texttt{histi} for BOX~2.
  \end{algorithmic}
\end{algorithm}
\section{History vector layout}
Table~\ref{tab:histi} summarises the entries that \texttt{stress\_dp\_bn} exchanges with the global loop. The values are written into the work array at the end of each call and read back during the next Newton iteration.
\begin{table}[H]
  \centering
  \begin{tabular}{@{}lll@{}}
    \toprule
    Slot & Symbol & Description \\
    \midrule
    1 & \(\delta\gamma\) & Consistency increment provided by BOX~2 (not overwritten locally). \\
    3 & \(\alpha\) & Updated equivalent plastic strain. \\
    4 & \(g\) & Yield residual after the BOX~1 update. \\
    5 & \(N\) & Scalar consistent modulus required in BOX~2. \\
    6 & \(g\) & Stored again for convenience when forming the residual vector. \\
    7--15 & \(\beps\) & Current total strain tensor components. \\
    16--24 & \(\partial g/\partial \beps\) & Sensitivity used when building the global coupling. \\
    25--33 & \(\bM\) & Flow-direction matrix used in the \(\bM^T\) block of BOX~2. \\
    34 & \(\Delta\gamma\) & Cumulative consistency parameter for monitoring. \\
    \bottomrule
  \end{tabular}
  \caption{Selected \texttt{histi} entries after calling \texttt{stress\_dp\_bn}.}
  \label{tab:histi}
\end{table}
\section{Residual coupling in BOX~2}
The global block Newton system couples the structural equilibrium residual \(\mathbf{R}_f\) and the yield residual \(R_g\):
\begin{equation}
\begin{bmatrix}
\mathbf{K}_T & \mathbf{L}\\
\mathbf{M}^T & N
\end{bmatrix}
\begin{bmatrix}
\delta \mathbf{u}\\
\delta\gamma
\end{bmatrix}
= -\begin{bmatrix}
\mathbf{R}_f\\
R_g
\end{bmatrix}.
\label{eq:block_system}
\end{equation}
The element contributions required in \eqref{eq:block_system} are available from \texttt{stress\_dp\_bn}: the consistent tangent \(\bC^{\mathrm{ep}}\) contributes to \(\mathbf{K}_T\), the matrix \(\bL\) (stored implicitly through \(\partial g/\partial \beps\)) contributes to \(\mathbf{L}\), \(\bM\) yields \(\mathbf{M}\), and the scalar \(N\) is accumulated into the lower-right block \cite{stressBNUpdate,stressBNHistory}. The element routine must contract these quantities with the strain--displacement matrix \(\bB\) when assembling the global system. During the following Newton iteration the solved increment \(\delta\gamma\) is fed back through \texttt{histi(1)} so that the integration-point update can proceed without local iterations \cite{stressBNHistory}.
\section{Practical usage}
To use the block Newton constitutive update in \texttt{plstss240}:
\begin{enumerate}
  \item Choose a material flag value (for example, reassign \texttt{MATYPE=4}) in the input data so that \texttt{phexa8} dispatches to \texttt{stress\_dp\_bn} instead of \texttt{stress\_dp\_rm} \cite{phexa8Call}.
  \item Ensure that the element assembly forms the mixed block matrix \eqref{eq:block_system} by integrating the additional \(\mathbf{L}\), \(\mathbf{M}\), and \(N\) contributions recorded in \texttt{histi}. The stored sensitivities map directly onto the required surface terms of Yamamoto et~al.'s BOX~2 formulation \cite{stressBNHistory}.
  \item During the global solve, populate \texttt{histi(1)} with the solved \(\delta\gamma\) increment before re-entering \texttt{stress\_dp\_bn} so that the integration-point update uses the globally consistent value \cite{stressBNHistory}.
\end{enumerate}
\bibliographystyle{plain}
\begin{thebibliography}{9}
\bibitem{analysNR} `src/analys.f`, lines~330--420.
\bibitem{postprDispatch} `src/postpr.f`, lines~166--205.
\bibitem{phexa8Call} `src/phexa8.f`, lines~160--214.
\bibitem{stressBNBox1} `src/stress\_dp\_bn.f`, lines~62--215.
\bibitem{stressBNUpdate} `src/stress\_dp\_bn.f`, lines~221--500.
\bibitem{stressBNHistory} `src/stress\_dp\_bn.f`, lines~506--548.
\bibitem{stressBNElastic} `src/stress\_dp\_bn.f`, lines~562--610.
\end{thebibliography}
\end{document}
